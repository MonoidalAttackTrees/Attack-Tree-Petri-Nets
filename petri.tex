General notes on petri net firing/reachable markings/safe/bounded
%%    A multiplicity function $V : F \to \mathbb{N}^+$, and
%%   A flow relation, $T \subseteq (P \times T) \cup (P \times T)$,
%% 	tokens: a discrete number of marks denoted as dots inside the places \\
%% 	these track the execution of the petri net - show what 'instruction' it is on, or what state it is in\\
Firing of a petri net demonstrates the flow of events through the net, from the initial marking through successor markings, based on the firing rules (given in F).\\
Pre-place: an input place; a place with an arc to a transition; denoted $(\ldotp , t) or  \cdot t$\\
If a pre-place has at least as many marks as the output arc's weight, then it is fulfilled and the transition the arc points to is activated, causing the transition to output to the post-place. However, the transition outputs the weight of its output arcs. Output $\ne$ Input\\
Post-place: an output place; a place with an arc from a transition; denoted $(t, \ldotp ) or t \cdot $ \\
A firing sequence from a place is a trace through a sequence of transitions\\
A reachable marking from one place to another means that there is a firing sequence from the first place to the last\\
The set RN := RN($m_0$) is the state space, the set of all reachable states in the petri net\\
This holds information about what events are possible and impossible in the petri net.\\
Bounded: a petri net is bounded if its state space is finite. \smallskip\\

A k-safe petri net is one in which, in every marking reachable from the initial marking, there are at most k tokens.\\
In classic petri nets, aka 1-safe petri nets, places can have at most one mark\\
 

