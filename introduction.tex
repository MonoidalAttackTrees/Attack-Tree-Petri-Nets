\par Attack trees provide a useful visual model to lay out all the possible avenues of attack upon systems or resources. These models have a number of advantages. Because they are fairly easy to understand, attack trees facilitate communication between professionals with different areas of expertise; such collaboration among specialists improves the quality of the security analysis. In addition, attack trees help defenders communicate with both detail and clarity, improving the quality of their end product. Attack tree models also encourage the people building them to decompose attacks into smaller steps; by helping to draw out lines of attack, these models lead to a more comprehensive view of both vulnerabilities and methods of defense. The attack tree method encourages defenders to take the attacker's view by drawing attention to the attacker's goals and the available means to achieve them, thus helping them to develop a more complete defense. Attack trees also provide good visual documentation of a team's collaborative efforts to discover and mitigate security vulnerabilities, facilitating their efforts to secure a system or resource. 
\par Given their many strengths, it is no surprise that attack trees have been a key tool for cybersecurity professionals since their development. However, as cybersecurity vulnerabilities and defenses become more complex, it becomes more difficult to use an attack tree to model them. One problem is that attack trees are generally built by hand; though there are tools to assist with the process, it remains labor-intensive, and the work required increases with the complexity of the system and its vulnerabilities. Beyond this, even in relatively simple scenarios, it is often possible to construct two attack trees that look very different, but in fact accurately capture the same vulnerabilities of the system. This problem of variation in equivalent attack trees also naturally increases as the complexity of the attack tree increases. As purely visual models, attack trees have very little formal structure to support them, and few resources to deal with the problems sketched above. However, recent work in the field has established a link between attack trees and Petri nets, which are both graphical and mathematical models.
\par Petri nets, first developed by Schneier (1999), are composed of places and transitions, with arcs connecting the two. The places model the state of the system under attack, and/or the progress of an attacker toward the goal. The transitions between the places represent actions that the attacker or the system can take to progress or repel the attack. The arcs link places to transitions, and transitions to places to show the paths through the net. Petri nets can be represented mathematically as well as graphically, allowing us to reason about them at a more abstract level while still maintaining an important common ground with attack trees. 
\par Researchers have recently begun to apply category theory to Petri nets. Category theory allows us to abstract even further, so that we can observe the most essential features of the objects under study. In stripping away the nonessential, category theory also helps us to notice important relationships between different objects that were not obvious before. By viewing Petri nets as a category, we can use abstraction and comparison to discover and prove important mathematical features that a purely graphical model like attack trees cannot reveal. MORE HERE
\par Our contribution to the current research in the field is the development of a chainable Petri net, or CNET, which tracks both its initial and final places (the beginning and end points of the net) to make it easy to chain these nets together to build a large, complex net out of simple pieces. To join the simple nets together, we have developed connectives based on logical operators frequently found in attack tree models, such as sequential composition, disjunction, and several varieties of conjunction. 
\par Ultimately, by joining in the effort to translate attack trees into mathematical/categorical models instead of simply visual models, we  hope to give cybersecurity professionals a model that is just as useful as attack trees, but that additionally has properties provable with mathematical certainty, such as equivalence or hierarchy between two nets.
    
