\begin{definition}
\label{def:Graph}
Graph G: a 4-tuple $(T, V, src, tar) where $\\ 
T is a set of edges; \\
V is a set of nodes;\\
src is $T \rightarrow V$ is the source of a given edge;\\
tar is $T \rightarrow V$ is the target of a given edge.\\
\end{definition}
%
\begin{definition}
\label{def:Graph-Morphism}
A graph morphism, h : $G \rightarrow G'$, between graphs is a  pair (f, g) where:\\
f: $T \rightarrow T'$ is a function, and\\
g: $V \rightarrow V'$ is a function.\\
\end{definition}
%
\begin{definition}
\label{def:Finite-Multiset}
A finite multiset over a set S can be defined as a formal sum:\\
$n_1a_1, \oplus ... \oplus, n_ia_i$\\
where $n_1, ... n_i \in N$, and $a_1, ... a_i \in S$\\
The formal sum obeys:
\begin{itemize}
\setlength\itemsep{.1em}
\item na: a occurs n times in multiset,
\item order doesn't matter,
\item na $\oplus$ ma = (m+n)a,
\item na $\oplus$ 0 = na = 0 $\oplus$ na.
\end{itemize}
\end{definition}
%
\begin{definition}
\label{def:Free-Commutative-Monoid}
The free commutative monoid generated by the set A is:\\
$A^\oplus$ =$\{$m $|$ m is a finite multiset over A$\}$.\\
\end{definition}
%
\begin{definition}
\label{place-transition-petri-net}
A place-transition Petri Net is a graph, $(T, S^\oplus)$ where T is called the set of transitions and S is called the set of places.\\
The elements of $S^\oplus$ model the number of markings required for a transition to fire at a particular place.\\
\end{definition}
%
\begin{definition}
\label{Monoid-Homomorphism}
A monoid homomorphism between two monoids $(M_1, \oplus_1, 0_1)$ and $(M_2, \oplus_2, 0_2)$\\
 is a function $f: M_1 \rightarrow M_2$ such that:\\
\begin{itemize}
\setlength\itemsep{.1em}
\item $f(a_1 \oplus_1 a_2) = f(a_1) \oplus f(a_2)$
\item $f 0_1 = 0_2$
\end{itemize}
\end{definition}
%
\begin{definition}
\label{Petri-net-morphism}
A Petri Net morphism between two Petri nets $(T_1, S_1^\oplus)$ and $(T_2, S_2^\oplus)$ is a graph morphism (f, g) where $g: S_1^\oplus \rightarrow S_2^\oplus$ is a monoid homomorphism.\\
\end{definition}
Thus, take as objects Petri nets $(T, S^\oplus)$ and as morphisms Petri net morphisms. This defines a category Petri.\\
\begin{lemma}
Suppose we have two Petri nets, $(S_1, \oplus, 0)$ and $(S_2, \oplus, 0)$.\\ 
\[(S_1 ^\oplus \times S_2 ^\oplus) \cong (S_1 + S_2)^\oplus\]
\begin{proof}
To show this, we need to define two homomorphisms between them, which we will call F and G.\\
\bigskip\\
It is important to note that the $\oplus$ operation is commutative. This commutativity allows the elements of the resulting free commutative monoid (FCM) to be rearranged, which will help us in our proof by allowing FCMs to be sorted in the following way:\bigskip\\
Consider an arbitrary $s \in (S_1 + S_2)^\oplus$. Since the generator is the disjoint union of $S_1$ and $S_2$, the elements of $s$ are drawn from those two sets, but are not grouped according to their origin set. Using commutativity, we can move all $y_i \in S_1$ to the front, and all $z_j \in S_2$ to the end. Then the form of $s$ is $(\oplus_i y_i) \oplus (\oplus_j z_j)$, where the elements are grouped according to their origin set. Any element of $(S_1 + S_2)^\oplus$ can be similarly sorted. For the purposes of this proof, assume that all such elements are so sorted.\bigskip \\
Now, let us define the function F: $(S_1 + S_2)^\oplus \rightarrow (S_1 ^\oplus \times S_2 ^\oplus)$.\\
Let k be an arbitrary element of $(S_1 + S_2)^\oplus$. We know that $k \in (S_1 + S_2)^\oplus$ must have the form $(\oplus_i y_i) \oplus (\oplus_j z_j)$, where $y_i \in S_1$ and $z_i \in S_2$ as described above. Furthermore, since the + operator is disjoint union, each element includes a marker indicating its origin set, which we can use to locate the boundary between elements of $S_1$ and elements of $S_2$. Therefore we can define F as follows:\\
\[F((\oplus_i y_i \in S_1) \oplus (\oplus_j z_j \in S_2)) = ((\oplus_i y_i \in S_1), (\oplus_j z_j \in S_2))\]
\[F((S_1, \oplus) \oplus (S_2, \oplus)) = ((S_1, \oplus), (S_2, \oplus))\]
This allows an element of $(S_1 + S_2)^\oplus$ to be matched with an element of $(S_1 ^\oplus \times S_2 ^\oplus)$, since the generator of $(S_1 + S_2)^\oplus$ contains the generator for both $S_1^\oplus$ and $S_2^\oplus$.\bigskip\\ 
Another important feature of FCMs arises from lifting the coproducts of sets to FCMs. For the coproduct of sets, we know that there exist the following functions:\\
\[inj_1: S_1 \rightarrow (S_1 + S_2) \]
\[inj_2: S_2 \rightarrow (S_1 + S_2)\]
Likewise, for the coproduct of FCMs, there are the functions: \\
\[inj^\oplus_1: S_1^\oplus \rightarrow (S_1 + S_2)^\oplus  \]
\[inj^\oplus_1(\oplus_i n_i x_i) = \oplus_i n_i(inj_1 x_i)\]
\smallskip
\[inj^\oplus_2: S_2^\oplus \rightarrow (S_1 + S_2)^\oplus \]
\[ inj^\oplus_2(\oplus_i n_i x_i) = \oplus_i n_i(inj_2 x_i)\]
Using this property, let us now define $G: (S_1 ^\oplus \times S_2 ^\oplus) \rightarrow (S_1 + S_2)^\oplus$.\\
Let $k$ be an arbitrary element of $(S_1 ^\oplus \times S_2 ^\oplus)$. We know $k$ must have the form $(M_1 , M_2)$ where $M_i$ is a multiset of the form $(n_1z_1 \oplus ... \oplus n_bz_b)$, and all $z \in S_i$ for $i\in\{1, 2\}$ (all n being natural number counters for the multiset).\\
Therefore we define the function G as follows:\\
\[G(M_1, M_2) = (inj_1^\oplus M_1) \oplus (inj_2^\oplus M_2)\]
By this transformation, we can match an element of $(S_1 ^\oplus \times S_2 ^\oplus)$ with an element of $(S_1 + S_2)^\oplus$, since by the nature of coproducts, any s $\in S_1^\oplus \in (S_1+S_2)^\oplus$, and similarly any s $\in S_2^\oplus \in (S_1+S_2)^\oplus$.\bigskip \\
Now that we have defined a homomorphism in both directions, we must prove that F;G = Id and G;F = Id. Take an arbitrary multiset M of the form $(n_1z_1 \oplus ... \oplus n_bz_b)$, with some elements drawn from a set $S_a$ and some drawn from $S_b$ (all n being natural number counters for the multiset). Sorting this multiset will yield the form $(\oplus_a n_a y_a) \oplus (\oplus_b n_b z_b)$, where $y_a \in S_a$ and $z_b \in S_b$. Putting this into the functions, we get:\\ 
\[F((\oplus_a n_a y_a) \oplus (\oplus_b n_b z_b)) = ((\oplus_a n_a y_a), (\oplus_b n_b z_b))\]
\[G((\oplus_a n_a y_a), (\oplus_b n_b z_b)) = ((\oplus_a n_a y_a) \oplus (\oplus_b n_b z_b))\] 
Thus, F;G = Id.\\
Now let us check for identity in the opposite direction, proving G;F = Id.\\
Take arbitrary sets $S_a$ and $S_b$. Generate the FCM of each and then their product: $(S_a^\oplus \times S_b^\oplus)$. The result will have the form $((\oplus_a n_a y_a), (\oplus_b n_b z_b))$, where $y_a \in S_a$ and $z_b \in S_b$. Putting this into the functions, we get:\\ 
\[G((\oplus_a n_a y_a), (\oplus_b n_b z_b)) = ((\oplus_a n_a y_a) \oplus (\oplus_b n_b z_b))\] 
\[F((\oplus_a n_a y_a) \oplus (\oplus_b n_b z_b)) = ((\oplus_a n_a y_a), (\oplus_b n_b z_b))\]
Thus, G;F = Id. 
Since $(S_1 ^\oplus \times S_2 ^\oplus) \leftrightarrow (S_1 + S_2)^\oplus$, we can conclude that $(S_1 ^\oplus \times S_2 ^\oplus) \cong (S_1 + S_2)^\oplus$.\\
\end{proof}
%
\end{lemma}
%
\begin{definition}
\label{Chainable-Petri-Net}
A chainable petri net is a Petri net N =($\delta_0, \delta_1: T \rightarrow S^\oplus$) with two additional features: an element i $\in S^\oplus$ that points to the initial marking, and an element f $\in S^\oplus$ that is the endpoint of the final marking.  $tar(i)=(s_1 \oplus s_2 ... \oplus s_n)$, with elements not duplicated: $s_j \neq s_k$ when $j \neq k$. i points to the initial marking of the net, and so it gives the starting point for the net. Thus, $src(i) = \emptyset$.\\
The element f tracks the ending of the net, so $src(\{x\in final\})= f$, and $tar(f) = \emptyset$. Morphisms for chainable nets must preserve both i and f. So, a morphism $<a,b>: (N, i, f) \rightarrow (N', i', f')$ is an ordinary net morphism that preserves the markings - b(i) = i' and b(f) = f'. \\
\end{definition}
The chainable petri net, or CNET, is a petri net that tracks both its initial and its final places, to facilitate chaining the petri net together with other petri nets at either end. This enables the use of operators such as SEQ, AND, \& OR.\\
Suppose we have two CNETs, $(N_1, i_1, f_1)$ and $(N_2, i_2, f_2)$. \\
$N_1 = (\delta_0, \delta_1: T \rightarrow S)$ and $N_2 = (\delta_0', \delta_1': T' \rightarrow S')$.\bigskip\\
SEQ for CNETs is the composition operation:\\
$N_1$ SEQ $N_2$ = $N_1$ ; $N_2$ = $N_3$, where $(N_3,  i_3, f_3)$ such that:\\
$N_1$ ; $N_2$ = $T_1 \cup T_2 \rightarrow S_1^\oplus \cup S_2^\oplus$;\\
$i_3 = i_1;$\\
$f_3 = f_2;$\\
$tar(f_1) = i_2$;\\
$src(i_2) = f_1$.\bigskip\\
AND for CNETs is the tensor product?(under construction):\\
%$N_1$ AND $N_2$ = $N_1 \times N_2$ = $N_3$, where $(N_3,  i_3, f_3)$ such that:\\
%$N_1 \times N_2$ = $T_1 \times T_2 \rightarrow S_1^\oplus \times S_2^\oplus$;\\
%$i_3$ is a new place;\\
%$src(i_3 \times \emptyset) = \emptyset;$\\
%$tar(i_3\times \emptyset) =\{i_1 \oplus i_2\};$\\
%$f_3$ is a new place;\\
%$src(f_3\times \emptyset) = \{f_1 \oplus f_2\};$\\
%$tar(f_3\times \emptyset) = \emptyset;$ \bigskip\\
OR for CNETs is the coproduct(p.126):\\
$N_1$ OR $N_2 = N_3$, where $(N_3,  i_3, f_3)$ such that:\\
$N_1 + N_2$ = $T_1 + T_2 \rightarrow (S_1 + S_2)^\oplus$;\\
$i_3$ is a new place;\\
$src(i_3) = \emptyset;$\\
$tar(i_3) =\{i_1 \oplus i_2\};$\\
$f_3$ is a new place;\\
$src(f_3) = \{f_1\};$\\
$src(f_3) = \{f_2\};$\\
$src(f_3) = \{f_1 \oplus f_2\};$\\
$tar(i_3) =\emptyset$\bigskip\\
\begin {proof}
$N_3$ is a coproduct if it satisfies the following conditions: $N_3$ must have morphisms $inj_1: N_1 \rightarrow N_1 + N_2$ and $inj_2: N_2 \rightarrow N_1 + N_2$ such that for any object R and morphisms $f_1: N_1\rightarrow R$ and $f_2: N_2\rightarrow R$, there is a unique morphism $f: N_1 + N_2 \rightarrow R$ such that $f_1 = f(inj_1)$ and $f_2 = f(inj_2)$.\\
\bigskip\\
Based on the definition given, the places of $N_3$ are the places of $N_1 \cup N_2$, with a new place $i_3$. Likewise, the transitions of $N_3$ are those of $T_1 \cup T_2$, with a new transition $(i_3, i_1 \oplus i_2)$. Lifting the coproducts of sets to CNETS will give us $inj_1$ and $inj_2$. We need to verify that these morphisms adhere to the requirement described above.\\
\bigskip\\
Let us define the morphism $f: N_1 +  N_2 \rightarrow R$. 
\end{proof}
\newpage
Example nets:
%
K = \\
$S_K :\{ a, b, c\}$\\
$T_K :\{t\}$\\  
$F_K (a, t) = 2$\\
$F_K (b, t) = 1$\\
$F_K (t, c) = 2$\\
$F_K (else) = 0$\\
%
$S_K^\oplus :\{ 2a \oplus 1b \oplus 2c\}$\\
$t = \{ 2a \oplus 1b \rightarrow 2c \}$\\  
$\delta_{0K} (t) = 2a \oplus 1b \}$\\
$\delta_{1K} (t) = 2c \}$\\
\smallskip
M = \\
$S_M :\{d,e\}$\\
$T_M :\{t'\}$\\  
$F_M (d, t') = 2$\\
$F_M (t', e) = 1$\\
$F_M (else) = 0$\\
%
$S_M^\oplus :\{ 2d \oplus 1e\}$\\
$t = \{ 2d \rightarrow 1e \}$\\  
$\delta_{0M} (t') = 2d \}$\\
$\delta_{1M} (t') = 1e \}$\\
\smallskip
$S_K^\oplus \times S_M^\oplus = (2a \oplus 1b, 2d) \rightarrow (2c, 1e)$\\
\smallskip
$(S_K + S_M)^\oplus = (\{1\} \times S_K) \cup (\{2\} \times S_M)^\oplus$\\
$((1, 2a), (1, 1b), (2, 2d))^\oplus = $\\
$S_{KM}^\oplus: \{(1, 2a), (1, 1b), (2, 2d) \}$\\
$T_{KM}: \{t_1, t_2\}$\\
$t_1 = ((1, 2a) \oplus (1, 1b)) \rightarrow (1, 2c)$\\
$t_2 = (2, 2d) \rightarrow (2, 1e)$\\
$\delta_{0KM} (t_1) = (1, 2a) \oplus (1, 1b) \}$\\
$\delta_{0KM} (t_2) = (2, 2d) \}$\\
$\delta_{1KM} (t_1) = (1, 2c) \}$\\
$\delta_{1KM} (t_2) = (2, 1e) \}$\\
\smallskip
$(S_1 ^\oplus \times S_2 ^\oplus)$ and $(S_1 + S_2)^\oplus$ are isomorphic. \\
Suppose there are two Petri nets $S_1 and S_2$, with sets of places $M_1 and M_2$ and sets of arcs $T_1 and T_2.$\\
Let F be the homomorphism from $(S_1 + S_2)^\oplus$ to $(S_1 ^\oplus \times S_2 ^\oplus)$. \\
$F((S_1 + S_2)^\oplus) = (S_1 ^\oplus + S_2 ^\oplus)$\\
where $S_1 + S_2 = (\{1\} \times S_1) \cup (\{2\} \times S_2)$\\
For F:\\
Calculating the set of places of the codomain:\\
  $S_1 =  \bigcup_{ y \in domain (x, y)} | x=1)$\\
  $S_2 =  \bigcup_{y \in domain (x, y)} | x=2)$\\
Note: I mean to separate out the two sets using the disjoint markers here.\\
The set of places of the codomain = $(S_1 ^\oplus \times S_2 ^\oplus)$\\
The arrows of the codomain maintain the structure defined in the domain, disregarding the origin markers \{1\} and \{2\} added in by taking the disjoint union.\\
Since the function preserves the structure between the objects, only changing the names of the objects, this is a homomorphism.\\
Now, let G be the homomorphism from $(S_1 ^\oplus \times S_2 ^\oplus)$ to $(S_1 + S_2)^\oplus$. \\
The objects or places of the codomain are the objects of the domain modified in the following way:\\
where objects in the domain have the form $(x, y), (\{1\} \times x) \cup (\{2\} \times y).$\\
The arrows of the codomain maintain the structure of the domain, adding in origin markers \{1\} and \{2\} by position as described above.\\
Since G preserves the group's structure and only modifies objects' names, it is a homomorphism. 
Clearly,$ G \circ F = F \circ G,$ since F removes the origin markings  \{1\} and \{2\} and G puts them back. Structure remains unchanged in either direction.\\
\newpage

% subsection MeseguerMontanari (end)
