%%%%%%%%%%
\section{Petri-Nets-Categorically}
In \cite{2018arXiv180805415B}, the authors define Open Petri nets as Petri nets with the addition of designated input and output places in order to study the reachability semantics of Petri nets, a way of calculating what outputs are possible given some inputs to a Petri net. They follow the work of Meseguer and Montanari in looking at Petri nets as symmetric monoids, but further develop the application of category theory to Petri nets by understanding Open Petri nets as a double category, which allows them to ???not sure exactly, need to read and think some more??? The authors' notion of an Open Petri net, with input and output places, is similar to our CNETs, though they are concerned with a different problem. We formulated CNETs to build up attack trees, whereas their concern is with reachability in a pre-existing net. Thus their work and ours is complementary; their reachability semantics is a powerful tool to analyze the CNETs we can build.
\par In \cite{MESEGUER1990105}, the authors take a category-theoretical approach to study concurrency in Petri net models. Making the point that Petri nets can be understood as directed graphs, they introduce a category for Petri nets generally, and also for pointed Petri nets, which have an initial marking to show the beginning of the net. They also work through the categorical product and coproduct for Petri nets, and show that Petri nets are monoids where the objects are the places and the relation is given by the transitions. Our work builds on theirs by continuing the categorical approach. We also built upon their pointed Petri net to define our CNET, which is a Petri net with both a start and end place to allow for chaining. They develop two operators for the pointed Petri net, $/oplus$ and $;$, which we further developed for our CNET operators AND and SEQ, respectively.  
\par In \cite{soton261825}, the author surveys his own contributions to algebraic, compositional approaches to the semantics of Petri nets, approaches that, broadly speaking, develop a view of Petri nets as complex structures built up from simple parts and a limited number of connection types, or combinators. The author follows the development of the field from its beginnings, starting with the application of process algebras and categorical algebra to Petri nets. In particular, his research has focused on concatenable Petri nets, and ultimately has followed Meseguer and Montanari, 1990 into the categorical approach to Petri nets (as monoidal categories). This paper provides a helpful and thorough history of ideas that our paper builds upon. We take a very similar approach to solving the problem of applying Petri nets to attack trees, but Sassone is not concerned with this specific problem. 
\par In \cite{SASSONE1996277} Sassone adds to the categorical understanding of Petri nets, building on the conceptualization of concatenable processes as morphisms (or what he calls "arrows") of a symmetric monoid. He discusses the complications of concatenating processes when done by merging the first net's maximal places (the final marking in our paper) with the next net's minimal places (what we call the initial marking). The difficulty arises because the place-to-place correspondence can be done in multiple ways, resulting in different concatenation behaviors. His solution to this problem is to focus on the tokens that move through the net, rather than the places.  Within this line of reasoning,  he points out that the category of the concatenable processes of a net are themselves a symmetric monoidal category with a tensor product yielding the parallel composition of processes, which in turn shows that Petri net behavior can be understood algebraically. He proposes to take the work of Deguano, Meseguer, and Montanari (1989) and make it completely abstract and axiomatic by establishing a category of symmetries of the Petri net category. While this research is not concerned strictly with what we are examining, it provides some helpful insights into concatenation and parallelization in Petri nets viewed categorically.
\par In \cite{BRUNI2001207}, according to the authors' interpretation of the field, the research into the semantics of place/transition Petri nets can be divided into two camps: collective and individual token philosophies. Whereas the collective side does not distinguish between different instances of tokens at the same place, reasoning that what really matters is whether a token occupies the place or not (i.e. for the firing of the next transition), the individual side does pay attention to the particulars of the token - especially as its identity relates to its causal history. The authors identify an issue with the algebraic semantics of place/transition Petri nets interpreted under the individual token philosophy, showing that it lacks universality and functoriality. To address this, they introduce the idea of pre-nets, whose states are strings of tokens (which are total orders) rather than the multisets we use in our work. However, our understanding of Petri nets is strongly based on pomsets as expressive of a particular path through the net, and so their pre-nets and the associated category-theoretical apparatuses might prove useful to us. Furthermore, our research falls more into the individual token philosophy, so our theory may be vulnerable to the weakness they describe and repair with the pre-net. This article may prove important in the theory of our paper, but our paper has a narrower application and develops the theory of Petri nets in a direction that they do not explore.
%%%%%%%%%%%%%
\section{Attack-Trees}
The seminal work in attack trees is \cite{Schneier1999}. In this paper, the author proposes using a tree as a way to model cyber threats. By clearly enumerating and describing the vulnerabilities of a given resource, he hopes to facilitate defense against those vulnerabilities. In attack trees, the goal is placed at the root, and the paths up the tree represent the different ways of achieving that goal. Like trees in general, the structure is hierarchical, with goals comprising subgoals, and in turn being part of larger goals. The child trees can be joined by AND nodes or OR nodes, with these carrying the usual meaning. He goes on to describe a simple calculation of "possible" or "impossible" paths, obtained by assigning a mark of either possible or impossible to each leaf. Then, if one child of an OR node is possible, the OR node is marked possible. Both children of an AND node must be marked possible for the parent to be possible. This calculation pattern is repeated up the tree until all paths are calculated, giving a rough idea of how secure a resource is. Schneier then demonstrates a similar process for calculating the cost of each path. He then shows how such calculations can inform cyber defense, for example by targeting the cheapest attacks first, or considering how to make a path more expensive to follow. Having established the utility of attack trees, the author goes on to explain how to construct them. This process breaks the single goal of compromising the resource into smaller and smaller sub-goals until the paths are fully developed and exposed. He also recommends collaboration in the construction of attack trees, to provide a fresh perspective to uncover all possible attacks. Then, he defines the strengths of the attack tree model, highlighting its ability to uncover unexpected vulnerabilites and its potential for reuse in similar situations or in the protection of similar resources. He also points out that the reuse feature means that professionals who lack expertise in a particular area can still mount effective defenses in those areas by relying on the expertise of those who made the pattern. 
\section{Modeling-Attack-Trees-as-Petri-Nets}
\par In \cite{LOUNIS2018135}, the authors are doing research very similar to our own. They recognize the value of attack-defense trees, a refinement of attack trees in which both attacks and defenses can be modeled, but also see the limitations of existing semantics for ADTrees. Thus, in this paper they develop a semantics based on stochastic Petri nets that can capture execution, order of steps, and dependencies. In stochastic Petri nets, transition firings are delayed by an assigned or randomized amount of time. Relying on the isomorphism between finite continuous time Markov chains and bounded stochastic Petri nets, they show how to quantitatively analyze ADTrees. And, using the established analysis methods for Petri nets, they show how to qualitatively analyze ADTrees. They also define a new operator for parallel execution of two branches in a Petri net. In their semantics, they distinguish a set of initial and final places for each stochastic Petri net (SPN), and then begin to construct an SPN from an ADTree by starting with the leaf nodes, building a single-transition net with an initial place and a final place. Conjunction they define as the Cartesian product of the conjuncts, where the initial places and final places are each paired to become the initial and final places of the conjunction, and then the places and transitions are unioned together. This is how we defined conjunction for CNETs, except that these authors have the added element of stochasticity. They define disjunction by merging the initial place of all disjuncts into one, and then allowing the final markings of each branch to count as the final marking for the disjunction. This again follows our construction.  They define sequential composition in a similar way, by merging the final place of the first with the initial place of the second. Their parallel disjunction operator brings together the initial places of each branch, but then requires only one branch to complete. \todoBA{I don't understand their notation for this, but if I'm reading it correctly I also don't understand why they did this - why parallelize disjunction, since by its very nature one disjunct doesn't care about any of the others?} They then develop semantics for countermeasures, and work through an example to show the effectiveness of their method. Our paper shares a lot of material with this paper; however, we do define other connectives that they do not have, such as parallel and synchronous conjunction. Also we approach the subject categorically and offer proofs of the properties of our connectives.  
\par In \cite{McDermott:2001:ANP:366173.366183}, the author provides a brief introduction into penetration testing to discover potential attacks on a system, and discusses two approaches: flaw hypothesis and attack tree. Flaw hypothesis analyzes a system within a particular scope, generates hypothetical flaws and then tests for them, using any confirmed flaws as a basis for generalization (to find similar flaw patterns in other areas of the system) and further testing. Finally, any confirmed flaws are eliminated. The attack tree approach uses a tree graph with the goal as the root, and then the steps to achieving the goal as the branches and/or leaves. In this paper, McDermott proposes organizing an attack tree more formally as a Petri net, with places representing states of the system, and transitions representing actions or events that cause a change in state. He describes the attack net approach as contributing several strengths to penetration testing: modeling concurrency and the current state of an attack with tokens, showing intermediate goals more clearly as places, and shoing the cumulative effects of different avenues of attack on the whole system. This is clearly a foundational article for our research. Besides establishing the central idea of modeling attack trees as Petri nets, this article also suggests important features for our system that we have worked to incorporate, such as using a flaw database to pattern-match attacks, and building up nets from simple attacks to more complex ones. Our work develops the central idea more specifically through our models of different connectives and our use of category theory, but this article is invaluable as a basis for our further efforts.
\par \todoBA{Make a bibtex entry for this one} Chowdhury, Farida and Ferdous, Md. Sadek. (2017). Modelling Cyber Attacks. International Journal of Network Security and Its Applications. 9. 13-31. 10.5121/ijnsa.2017.9402.  In this article, the authors aim to develop a model of cyber attacks which they claim is more comprehensive because it accounts for features of cyber attacks that other models do not include. They break the attack into three components: the attack itself, the target at which it is directed, and the victim (the entity that controls or otherwise is concerned with the fate of the target). Each attack is either visual, network-based, or a hybrid. Following this, they label each attack with the layer of the TCP/IP Protocol (or DARPA model) that it targets. They also distinguish between active attacks, where the target can be modified, and passive attacks, where the target can merely be observed. The authors codify several possible motivations an attacker might have, and develop taxonomies of possible attacks based on these motivations and general vulnerabilities a system might have. They gather each of these characteristics into a tuple of the form ***see picture to get Latex right***. While the authors are working to develop a mathematical model that captures important features of an attack, their model is essentially qualitative in nature, meaning that it is more of a classification system than a truly mathematical model. Our model is more detailed and allows for more effective analysis and comparison as a result. That being said, the model presented here does provide a helpful, detailed mechanism for general attack classification. 
\par In \cite{city8206,}, the authors apply Petri nets as a model for attack trees, identifying a major weakness in using this model: the labor-intensive process of translating a visual attack tree into a mathematical Petri net. They propose to resolve this difficulty by creating a large Petri net out of many smaller Petri nets, all created by different people with different areas of expertise. The joining of one Petri net to another is accomplished by matching a place in one Petri net with its identical place in another Petri net.\todoBA{need to read more to understand how they establish identity} While their approach is, on its face, very similar to ours, ours provides a more effective solution because it relies on attack patterns, which are more abstract. This facilitates reuse of our component trees, reducing error and labor. It also makes it possible to analyze of the overall structure for equivalence and hierarchy, which adds important functionality for cybersecurity professionals.

In \cite{1652085}, the authors describe the structure of attack trees and survey earlier attempts to represent them mathematically in (Mauw and Oostdijk 2005) and (Linger and Moore 2001). They also discuss using XML files to represent attack trees textually, in outline form, to facilitate automated analysis and simulation. They point out flaws with the attack tree model, particularly the lack of a standard form for the model, which hinders attack tree reuse. Then, they give an overview of Petri nets and show several behaviors that the Petri net can capture: concurrency, conflict (limitation and competition regarding resources), locality (some transitions only affect one area of the net, not the entire thing), boundedness (limitation on the number of tokens a place can hold), and liveness (no transition will become permanently unfireable). Moving to timed Petri nets, the authors explain that stochastic Petri nets provide random delay factors for all transitions in the net. They use these to simulate the net's runs and analyze the probablility of various markings. They then describe a method for modeling attack trees as Petri nets, describing conjunction and disjunction patterns. However, their patterns lack the intital and final places of our CNETs. Also, they explain that implementing the conjunction requires consideration of the number of tokens in the net, since a pair of tokens is needed at once to activate classical conjunction. They offer sequential conjunction as a workaround for this, but in fact this is a conflation of two different kinds of conjunction, which captures two different behaviors. Also, building a Petri net model of an attack tree in their system requires a lot of thought, because their small-scale connectives are not chainable. 

In \cite{KORDY20141}, the authors provide an survey of directed acyclic graphical models for security. Some notable models they include are attack trees and various extensions, Bayesian attack/defense graphs, and Boolean logic driven Markov processes. Clearly, their scope in this paper is wide. They give an account of Petri nets in security-related applications as well. They point out that Petri nets have many specializations and extensions that allow them to be used in a variety of contexts, and provide bibliographic information to demonstrate that fact. They survey existing literature on applications of Petri nets
%%%%%%%
\section{Workflow-Nets}
In \cite{2013CaCA}, the authors describe a type of Petri net called a workflow net that is very similar to our chainable Petri net. We did not discover this subclass of Petri nets before defining our CNETs because workflow nets are used to model processes and workflows (as the name implies) in areas such as business and manufacturing, whereas our focus has been in security. Like our CNETs, workflow nets have their initial and final places marked, and every node is on a path between the initial and final places. Also, every transition must be live, meaning that it has to be enabled at some reachable marking. The last two requirements are not explicit in our CNETS, but seem to be a natural occurence anyway given the way we aggregate small CNETS into bigger ones, and the general premise that we are mapping paths to a goal. How much CNETs will adhere to these requirements needs further study, as does the question of how closely process trees match attack trees. Interestingly, in this article the authors describe a process of translating a workflow net into a process tree, moving in the opposite direction from our research. Still, as a kind of dual to our research, this can offer important insights. The authors define the connectives of sequential composition, exclusive choice, and parallel composition using graphs, without the precision of defining each element in the Petri net tuple as we do in our paper. They omit these details because their focus is on the translation to process trees. They show that a workflow net is convertible to a process tree if it is acyclic, has no bridges, and is well-structured. It must be acyclic because the net must be able to complete in order to become a tree, since the tree structure includes completion as its structuring premise. Well-structuredness means that two paths coming off of a transition should not meet up in a single place, and two paths coming off of a single place should not meet up in a single transition. The authors then give an algorithm for translating a workflow net into a process tree. By evaluating the number of paths into and out of a node, together with the type of node (transition or place) they can infer the structure of the process tree, which their algorithm outputs as a formula in prefix notation. While this paper does cover similar ground to our research, our work still makes important additional contributions to the field because it applies a similar concept to a new area (cybersecurity) where it can be tremendously useful, and also makes the construction/aggregation of CNETs mathematically precise.  
\par 

\nocite{*}
\bibliographystyle{plain}
\bibliography{references}


\end{document}